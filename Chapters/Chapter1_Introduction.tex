Falls are one of the leading causes of fatal and non-fatal injuries especially for elderly people in today's society. Quick response is vital to reducing the long-term effects on the victim physically, emotionally and mentally. Falls have been one of leading cause of injuries across all age groups placing a strain on health and financial systems around the world. \\
As a result, an efficient, accurate and cost effective detection system needs to be implemented for a variety of environments. Over 50\% of falls for the 65+ age group (1/3 experience a fall at least once a year) occur in their home so a solution that is \textbf{cheap, accurate, privacy-focused} and \textbf{easily installed} is needed. \\
Personally, I have seen the effects falls can have on an elderly person and the fear induced in a fall victim post-treatment can deem them unable to live alone again. \\
There are a number of solutions on the commercial market today which can be divided into a broad number of categories: Wearables, Visual, Ambient Environment. Products on the market in these categories include the Apple Watch, cameras, accelerometer belts, floor vibration sensors, infrared devices and many more. As summarised in my Interim Presentation, each of these has their disadvantages. \textbf{Privacy} is a key issue that needs to be addressed across the world in the next decade. Therefore, it was a key aim of my project to keep it as privacy oriented as possible. Many of the current solutions on the market suffer from privacy issues especially in such a sensitive environment as somebody's home. Many of these off the shelf solutions require \textbf{Direct Line of Sight (LOS)} to the person in the room to detect a fall. No obstacles can be in the way of the detection apparatus or a fall may not be detected by the system. This is especially useless in a busy household environment with many obstacles. All of these existing solutions are \textbf{very expensive} to buy and implement such as the Apple Watch, specialist cameras and Man-Down alarms. Another issue arises from the wearable nature of a lot of these devices. Elderly people are not inclined to wear them as they hinder them from daily activities and can feel like a chore to keep them charged/around their neck or waist. This results in a lot of people refusing to wear them which is clearly unsatisfactory as a solution. More complex solutions such as Computer Vision cameras require \textbf{high processing power} as the calculations and classification operations needed for the sheer amount of data recovered in any of these fall detection experiments. \\
Therefore, I propose a Wi-Fi based solution which could utilise freely available commercial off the shelf Wi-Fi network cards and Access Points and provide a cheap, highly accurate and non-intrusive fall detection system for the home and other environments. A Fall can be detected due to the scattering in the rich multi-path environment that exists between a number of transmitting and receiving antennas. Previous research papers have explored the use of the Channel State Information (CSI) of a WLAN Channel for fall detection using commercial off-the-shelf devices. A changing channel can indicate that a fall has occurred privately and accurately so that the relevant authorities can be notified. \\
The aims of the project are as follows: 
\begin{itemize}[noitemsep, topsep=0pt]
\item Gain a strong understanding and background of CSI in the \textit{IEEE802.11n} Wi-Fi standard 
\item Build an initial system which can collect CSI using the Intel Wi-Fi Link 5300 NIC for a range of Transmitter-Receiver setups
\item Obtain CSI data for a range of human activities which would be typical in home/workplace environments
\item Design, build and test signal processing algorithms to clean the CSI data obtained for fall detection under various conditions
\item Design, implement and test various Machine Learning algorithms for fall detection for a target fall detection rate of $>$90\%
\end{itemize}
\begin{comment}
\begin{center}
\begin{tabular}{|c||p{10cm}|}
 \hline
 \multicolumn{2}{|c|}{\textbf{Project Timeline}} \\
 \hline
 \bfseries Date & \bfseries Task to be completed \\
 \hline
 19/09/19 & Allocation of Projects \\
 23/09 - 14/10 & Research OFDM, 802.11n, CSI \& Current State of the Art \\
 14/10 - 28/10 & Gather CSI data using Intel NIC under various activities, conditions \& environments \\ 
 28/10 - 18/11 & Researching, developing \& implementing signal processing algorithms to clean the CSI data for feature extraction \\
 26/11 & Interim Presentation \\
 27/11 - 6/12 & Gather data under more use cases \\
 03/01/20 - 15/01/20 & Identify features for activity segmentation of CSI data \\
 15/01 - 19/01 & Obtain much larger CSI datasets ($>$15mins of data) \\
 20/01 & Interim Report submission \\
 22/01 - 29/01 & Implement Activity Segmentation methods \\
 01/02 - 28/02 & Research, design \& implement different classification methods for fall detection \\
 29/02 - 22/03 & Test and tune classification methods for optimal fall detection accuracy \\
 27/03 & Conference Paper and Critique submission \\
 24/04 & Final Project Report Submission \\
 \hline
\end{tabular}
\end{center}
\end{comment}