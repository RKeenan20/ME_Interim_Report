\textbf{the code for the results came from the amplitudeSolverANNA.m file. The cleaner code for the .dat files is in ME Project CSI Info/22nd November Data/updated5GHz \\ csi5G\_CH36 etc in }
\section{Initial CSI Data Gathering}
The first section of my results will show the steps involved in obtaining the first CSI data file from the Linux Shell commands to the files I obtain at the end \par
\textbf{Add in 2 pictures of the data being gathered into the file on the Linux Terminal}
Once the CSI data was converted to absolute units as spoken about earlier, I was able to present one packet's SNR and CSI-phase across the 30 subcarriers.
Here, I present the difference between the data received at $2.4GHz$ by my smartphone and $5GHz$ by the Archer C6 router. I can see that both plots are very similar. Of course at this stage, there is not enough information to understand anything about the surrounding environment as I am only looking at one ICMP packet received. The CSI packet's data can be simplified as mentioned into \textit{streams}. (See \textbf{EQUATION STREAMS}. Hence, I can present each stream in terms of one subcarrier simplifying the problem as shown: \textbf{SHOW DIAGRAM OF one packet with all streams and then each stream having one value per packet}
Previous papers had found that the subcarriers and different streams are affected by human activities differently. I plotted the CSI amplitude of one subcarrier in different streams over time/packets. I also plotted the CSI subcarrier variances in one stream. \textbf{REFERNCE THE PLOT} From these plots, it is clear that for a human activity such as a fall, affects the different streams \textbf{differently} and the subcarriers \textbf{similarly}. 
\section{Data Processing}
\subsection{CSI Amplitude}
%%%%%%%%%%%%%%%%%%%%%%%%%%%%%%
To simplify each stream, I plotted the correlation matrix of the CSI subcarriers and found a strong positive correlation and high ratios between successive subcarriers. Thus, I decided to aggregate the subcarriers using \textbf{USE THE FORMULA FOR SUBCARRIERS}
However, I found that the amplitude variance between some immobile activities is quite high such as standing compared to sitting. Examining the difference between mobile and immobile activities is obvious as the signal variance in modified CSI amplitude clearly changes dramatically. It is clear that something dramatic has occurred in the environment but it is hard to notice a pattern or difference between some movements such as walking and falling.  \par
Other projects had also noticed how influential frequency domain fading was in their experiments. The Figure \textbf{REFERENCE FIGURE} demonstrates frequency domain fading in the channel. I can note for each stream or transmitter-receiver pair which ones demonstrate the most attenuation across the channel bandwidth, which suffer the least and which demonstrate the largest variation across the frequency band. 

\textbf{, we will be able to.....perhaps demonstrate LOS and NLOS and an extreme multipath environment on the data. }
\subsection{CSI Phase}
In terms of the CSI phase, I have noticed that it is a more granular indication of whether someone has fallen in the environment or not. As mentioned, I can plot the phase difference instantly as noted in PhaseU \citep{PhaseU}. These are shown below:
\textbf{PLOT PHASE DIFFERENCE FOR A NUMBER }
These plots contain the unknown phase offset and timing offset as alluded to in Equations \ref{eqn:3.3} \& \ref{eqn:3.4}. It is clear that it is difficult to recognise any such patterns for fall detection in the data. The linear transform removing the random noise, outliers and the phase offset as in Equation \ref{eqn:3.5}, I can present a much cleaner plot as shown below:
\textbf{PLOT THE CLEANER LINEAR PHASE TRANSFORM}
The plot above has shown that the phase offset, outliers and noise contribute a lot to the overall data and how we can recognise patterns in the data now. Hence, I can present a number of different plots with different human activities to see how the phase and amplitude respond. 
\textbf{PHASE AND AMPLITUDE RESPOND PLOTS}
\subsection{Phase Difference \& Amplitude for Fall Detection}
Obviously, previous papers had considered either the amplitude or phase for fall detection. In this section, I performed a number of experiments for my interim presentation which could put forward the idea that the phase difference was a much better indicator. These plots are shown on the same axis in terms of number of packets received (as the interpolation was not completed for the time domain analysis). One of the experiments is shown below for a 4m distance between Tx and Rx with myself as the test subject walking between the antennas before standing still, falling in place and getting up again. The results are shown:
\textbf{PLOT FROM MY INTERIM PRESENTATION}
As we can see from the experiment results, both the amplitude and phase experience quite a sudden disturbance on startup of monitoring. This is a combination of the rate selection algorithm of the Archer C6 router adjusting and the test subject walking around behind the Transmitter creating a multipath environment. Walking creates a mild disturbance in the middle of the experiment before the test subject falls creating a very large disturbance in the phase difference as shown. In terms of the CSI amplitude, we can see that the fall creates a larger disturbance than walking but a pattern isn't as noticeable. This can be attributed to the diminishing frequency diversity of the subcarriers due to the simplification in Equation \ref{eqn:3.1}. It is reasonably clear on the phase difference plot where the human activity finishes save for a few oscillations in the data. This will be very important for the future work of activity segmentation and feature extraction. As I have found, the phase difference is much more sensitive to human movement, even chest movement due to breathing, and as a result, it is much more suitable for fall detection. The CSI amplitude is a useful indicator for activity detection or determining whether somebody has done "something" in the environment. In a more complex MIMO-OFDM system (which is part of the future work), the phase difference resolution will improve due to the introduction of more antennas into the system on both the Tx and Rx side and I expect the performance to improve greatly for activity segmentation and feature extraction across LOS/NLOS environments. I will now consider the future work required to complete this project successfully. 