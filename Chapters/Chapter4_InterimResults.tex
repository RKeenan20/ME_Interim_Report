\textbf{the code for the results came from the amplitudeSolverANNA.m file. The cleaner code for the .dat files is in ME Project CSI Info/22nd November Data/updated5GHz \\ csi5G\_CH36 etc in }
\section{Initial CSI Data Gathering}
These are the results of gathering initial CSI data using Linux shell commands.\par
\textbf{Add in 2 pictures of the data being gathered into the file on the Linux Terminal}
I was able to present one packet's SNR and CSI-phase across the 30 subcarriers once the data was converted to absolute units as shown
I can see that both plots above are very similar. There is not enough information to understand anything about the surrounding environment as I am only looking at one ICMP packet received. With the CSI packet's data simplified into one value per stream using Equation \ref{eqn:3.1}, I can present each stream over a number of packets as shown below:  \textbf{SHOW DIAGRAM OF one packet with all streams and then each stream having one value per packet}.
To prove that human activities affect streams and subcarriers differently, I present the following:
\textbf{I plotted the CSI amplitude of one subcarrier in different streams over time/packets. I also plotted the CSI subcarrier variances in one stream.} \textbf{REFERNCE THE PLOT} From these plots, it a clear falling affects the different streams \textbf{differently} and the subcarriers \textbf{similarly}. 
\section{Data Processing}
\subsection{CSI Amplitude}
%%%%%%%%%%%%%%%%%%%%%%%%%%%%%%
To simplify each stream, I plotted the correlation matrix of the CSI subcarriers and found a strong positive correlation and high ratios between successive subcarriers. This proves I can aggregate the subcarriers as in Eqn. \ref{eqn:3.1}. This is shown below:
\textbf{PLOT SOME OF THE Variance for moving and not moving} 
As the plots above show, I found that the CSI-amplitude variance between some immobile activities is quite high such as standing compared to sitting. The difference between mobile and immobile activities is obvious as the signal variance in the modified CSI amplitude clearly changes dramatically. It is clear that something dramatic has occurred in the environment but it is hard to notice a pattern or difference between some movements such as walking and falling.\par
Other projects had also noticed how influential frequency domain fading was in their experiments. The plot above demonstrates frequency domain fading with most attenuated, least attenuated and highest variance subcarriers shown. 
\textbf{, we will be able to.....perhaps demonstrate LOS and NLOS and an extreme multipath environment on the data. }
\subsection{CSI Phase}
I have noticed that CSI phase is more granular indication of a fall in an environment. I plotted the phase difference, as noted in PhaseU \citep{PhaseU}, below:
\textbf{PLOT THE UNCLEAN PHASE DIFFERENCE FROM MY PRESENTATION }
These plots contain the unknown phase offset and timing offset as alluded to in Eqn. \ref{eqn:3.3} \& \ref{eqn:3.4}. It is clear that it is difficult to recognise any such patterns for fall detection in the data. Implementing Equation \ref{eqn:3.5}, I can present a much cleaner plot as shown below:
\textbf{PLOT THE CLEANER LINEAR PHASE TRANSFORM WHICH IS CLEANER FALL.DAT}
The plot above has shown that the phase offset, outliers and noise contribute a lot to the overall data and how we can recognise patterns in the data now. Hence, I can present a number of different plots with different human activities to see how the phase and amplitude respond. 
\textbf{PHASE AND AMPLITUDE RESPOND PLOTS}
\subsection{Phase Difference \& Amplitude for Fall Detection}
In this section, I performed a number of experiments for my interim presentation which could put forward the idea that the phase difference was a much better indicator. These plots are shown on the same axis in terms of number of packets received (as the interpolation was not completed for the time domain analysis). 
\textbf{CHANGE TO CAPTION=>One of the experiments is shown below for a 4m distance between Tx and Rx with myself as the test subject walking between the antennas before standing still, falling in place and getting up again.} The results are shown:
\textbf{THE LAST PLOT FROM THE PRESENTATION WHICH IS THE 2 SUBPLOTS}
Both the amplitude and phase experience a sudden disturbance on start up. This is a combination of the rate selection algorithm of the Archer C6 router adjusting and the test subject walking close to the receiver creating a multipath environment. Walking creates a mild disturbance in the middle of the experiment before the test subject falls creating a very large disturbance in the phase difference as shown. In terms of the CSI amplitude, we can see that the fall creates a larger disturbance than walking but a pattern isn't as noticeable. This can be attributed to the diminishing frequency diversity of the subcarriers due to the simplification in Eqn. \ref{eqn:3.1}. It is reasonably clear on the phase difference plot where the human activity finishes save for a few oscillations in the data. The CSI phase difference is more sensitive to human movement, even chest movement due to breathing, and as a result, it is much more suitable for fall detection. The CSI amplitude is a useful indicator for activity detection or determining whether somebody has done "something" in the environment. In a more complex MIMO-OFDM system, the phase difference resolution will improve due to the introduction of more antennas into the system and thus, performance for LOS/NLOS environments will greatly improve.