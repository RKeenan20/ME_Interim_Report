The future work required in this project from now until submission is comprised of a number of slight improvements to previous methods and the completion of the full fall detection framework. 
%\vspace{-33pt}
\section{Signal Pre-processing}
Up until now, I have researched and implemented many different pre-processing techniques which has led to the CSI amplitude and phase data being more interpretable to the untrained eye. As mentioned earlier, the transmitter and receiver do not attempt to synchronise in terms of timing and as a result, the time between received packets is not the same each time. This presents two issues: The sampled CSI data may not be continuous which is needed for feature extraction during a fall and I am unable to obtain a spectrogram of the data if they are unevenly spaced in the time domain. Without performing an operation to prevent this, feature extraction is extremely hard and inaccurate. To do this, I will implement a \textbf{ 1-D linear interpolation algorithm} the raw CSI. This is easily done in MATLAB using the \lstinline{interp1} function on each transmitter-receiver pair matrix ($CSI_{ij}$). This will allow me to see accurate receiving times for each packet for activity segmentation and feature extraction. \par
To make the training of my classifiers quicker, I can perform band-pass filtering on the interpolated CSI data. This can filter out the lower frequency components as seen by RT-Fall in the range [0,4]$Hz$ as most fall or fall-like activities occur in the range [5,10]$Hz$. 
\section{Activity Segmentation}
Firstly, for accurate and robust activity segmentation, I will need to gather a large amount of real-life CSI data from an indoor environment such as a living room, classroom or bedroom where falls are most likely to happen for elderly and other ages alike. I am currently in the process of carrying out this process to gather data of over 10+ minutes in duration where it will be harder to spot the pattern of a fall in CSI data especially at a high sampling/packet rate of $100pkts/sec.$. I need to find the finishing point of a fall or fall-like activity which I will do using a \textbf{threshold-based sliding window} firstly. To do this, I will need two stable signal streams across multiple sliding windows. From here, I can calculate their mean $\mu$ and normal standard deviation $\sigma$. I then calculate if an equation involving these is less than a threshold value. From here, I obtain two signal streams and detect if they are in a stable or fluctuating state. \par
The next stage of this is to determine the transition from a fluctuating to stable state. When there is a transition, I can mark this time as the start and the time of another transition as the end which gives me an activity window. From here, I can segment the activities by determining the appropriate trace back window size. 
\section{Feature Extraction \& Classification}
As seen in RT-Fall and WiFall which use similar number and type of features for recognising a fall, I will need to gather and research further features that could be an indication of a fall occurring. The vast majority of the projects and papers I have reviewed have used the same features. As a result, I will try to see if my workings are consistent with theirs and if there are any other features I could use. All of these features will form the input to my classifiers for training the associated model. \par
I will firstly design a binary classifier such as a simple SVM for detecting the different between a fall and a fall-like activity as these are the most similar and could lead the highest missed/false detection rate. The training dataset will need to be created in a usual living room. This will vary in terims of activities and durations of these activities. The activities will need to be segmented and labelled and then passed into the SVM classifier along with the extracted features. Through trail and error, errors in classification can be relabelled to improve accuracy. From here, I can provide a test set to check the accuracy after the model has been adjusted correctly. I will then implement a similar method using a Decision Tree Classifier such as RandomForest and perhaps, a Neural Network, if time allows and classification accuracy from SVM is not high enough. \par
Currently, I am in the process of extensive \& prolonged CSI data gathering for the activity segmentation step and the enhanced data pre-processing involving interpolation and filtering. 