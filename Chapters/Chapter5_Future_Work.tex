\section{Signal Processing}
Up until now, I have researched and implemented many different pre-processing techniques which has led to the CSI amplitude and phase data being more interpretable. The lack of synchronisation between Tx and Rx as mentioned in Section 2.2.3 presents two issues: The sampled CSI data may not be continuous which is needed for feature extraction during a fall and I am unable to obtain a spectrogram of the data if they are unevenly spaced in the time domain. Feature extraction is extremely hard and inaccurate with this issue. To resolev this, I will implement a \textbf{ 1-D linear interpolation algorithm} the raw CSI. This is done in MATLAB using \lstinline{interp1} on each transmitter-receiver pair matrix ($CSI_{ij}$). This will allow me to see accurate receiving times for each packet.\par
To make the training of my classifiers easier, I can perform band-pass filtering on the interpolated CSI data. This can filter out the lower frequency components as seen by RT-Fall in the range [0,4]$Hz$ as most fall or fall-like activities occur in the range [5,10]$Hz$ \citep{RTFall}.
\section{Activity Segmentation}
Having obtained real-life CSI data from an indoor environment of over 10mins to mimic a real test environment such as a bedroom, I need to find the finishing point of a fall or fall-like activity in this large amount of CSI data. I will use a \textbf{threshold-based sliding window} firstly. I will need two stable signal streams across multiple sliding windows. I can calculate their mean $\mu$ and normal standard deviation $\sigma$. I can calculate if an equation involving these is less than a threshold value. Obtaining two signal streams and detecting their state (stable/fluctuating) leads me to the next steps. \par
When there is a transition in signal state, I can mark this time as the \textit{start} and the time of another transition as the \textit{end} which gives me an activity window. From here, I can segment the activities by determining the appropriate trace back window size. 
\section{Feature Extraction \& Classification}
As seen in RT-Fall and WiFall which use similar number and type of features for recognising a fall, I will need to gather and research further features that could be an indication of a fall occurring. All of these features will form the input to my classifiers for training the classification model. \par
I will firstly design a binary classifier such as a SVM for detecting between a fall and a fall-like activity as these are the most similar activities and could lead the highest missed/false detection rate. The training dataset will need to be created in a usual living room. The activities will need to be segmented, labelled and passed into the SVM classifier along with the extracted features. With user feedback, classification errors will be relabelled to adjust the model. I will then test the model again with unseen data. \par 
I will then implement a similar method using a Decision Tree Classifier such as RandomForest and perhaps, a Neural Network, if time allows and classification accuracy from SVM is not high enough. 
\begin{comment}If I have enough time, I would like to investigate if the CSI.dat files are populated in real time and if there is a possibility of creating a Real Time system out of my work. \par \end{comment}
Currently, I am in the process of extensive \& prolonged CSI data gathering for the activity segmentation step and the enhanced data pre-processing involving interpolation and filtering. 