I will describe the technical work completed to date in order of completion and present any results obtained before moving on to future work to be completed. 
\section{Installing the CSI tool}
The first step was to install an Intel Wi-Fi Link 5300 Wireless NIC into a suitable laptop. The card itself has a mPCIe connection and many computers with this connection for their NIC are discontinued. They now use the newer M.2/NGFF connection. I installed the Intel NIC into a Toshiba Satellite L750 laptop connecting two antenna cables to the card (two antennas in Toshiba laptop). At the same time, I purchased a mPCIe to PCIe adapter which could be installed into a desktop computer. This was installed into a computer in the Project Lab in UCD with all three antennas connected for another Intel NIC. The Linux CSI tool is built to work with the Intel Wi-Fi Link 5300 using customised Intel firmware and customised versions of the \textit{iwlwifi} open-source driver \citep{Halperin_csitool}. \par
To install the Linux CSI tool, \cite{Halperin_csitool}, I first loaded Ubuntu 14.04 LTS on both the laptop and desktop. 
Following the instructions on the tool's website and reloading the \textit{iwlwifi} driver each time the laptop turns on, enables the CSI gathering functions. Running the \textit{log\_to\_file} command as instructed on the tool's README while also sending ICMP (ping) requests to an Access Point or Associated Router (without encryption) leads to CSI data being gathered. The CSI data is added to a .dat file whereby the CSI matrices are of the form as mentioned in Section \ref{chapter:Literature Review}. \par
I used my smartphone's Wi-Fi hotspot initially to verify the tool worked correctly. The phone's $2.4GHz$, one antenna, 802.11n Wireless card was able to transmit data at 802.11n rates. This created a $1\times 2$ SIMO network. I then tested the CSI tool with a commercial Virgin Media Access Point at home with three antennas and a $20MHz$ bandwidth. I demonstrated that the tool did not work with an access point with security enabled but it worked successfully with an access point not connected to the internet, thus keeping the network safe. This is due to not enough code space on the NIC for both beamforming software paths and encryption software paths which means CSI data can only be retrieved from non-encrypted access points.  \par 
I purchased a TP-Link Archer C6 (AC1200) which I could use without an Ethernet connection to act as a Transmitter for more antennas/customisability. This would allow me to perform experiments at $2.4GHz$/$5GHz$, Channel Bandwidth of $20MHz$/$40MHz$ and a range of WLAN Channels. The laptop/receiver would act as a receiver creating a $2\times 2$ or $2\times 3$ MIMO network. I attempted to set up \textit{hostapd} on both the desktop and laptop but unfortunately, I found that the Intel NIC was geolocation locked to not allow AP (Access Point) mode at $5GHz$ in its EEPROM which I was unable to change. As I would need the CSI phase at $5GHz$, I abandoned this work direction. 
\section{Collection of CSI Data}
CSI data is gathered when the client (laptop) with the modified \textit{iwlwifi} drivers reloaded is connected to an un-encrypted AP. In the Ubuntu terminal, the \textit{log\_to\_file} command is run. In a new terminal tab, a ping command is entered using the AP's gateway IP address. To maintain compatibility with other projects, I implemented a packet/sampling rate of \textit{100pkts/sec.}. The command to send an ICMP request to the AP from the client computer is \lstinline{sudo ping -i 0.01 -s 2000 <AP IP Address>}. When the packets are transmitted back to the receiver at HT/802.11n rates, the CSI tool command logs the data to the accompanying .dat file as shown in \textbf{REFERENCE THE RESULTS}
\subsection{MIMO Data Collection}
In terms of the $2\times 2$ and $2\times 3$ MIMO wireless networks using the Archer C6 router, I set the experiment up in a variety of environments usually separating Transmitter and Receiver by 4-5m. Both transmitter and receiver were located in a laboratory/classroom environment to mimic the intended locations as closely as possible. I needed useful data for a variety of activities for the data pre-processing stage of the system. I initially started with separate CSI data files where each activity was performed separately. Many human activities were carried out to see if there was any patterns in the CSI data such as standing, walking, sitting down, falling and standing still like RT-Fall. At this initial stage I was only concerned with identifying patterns in the CSI data. 
\begin{comment}In the future, I will need to devise an experiment which gathers 10+ minutes of real-life data from an environment such as a living room to identify consecutive human activities. This would provide a more accurate representation of how the channel changes due to consecutive human movements and lead to a more robust data processing and activity segmentation process.\end{comment}
\section{CSI Data in MATLAB}
I used MATLAB to process and analyse the obtained CSI.dat file. From here, I needed to unpack the binary format of the CSI data file using some of the provided scripts/MATLAB files by the tool's creators \citep{Halperin_csitool}. To unpack the binary data, I used a \textbf{MEX-file} compiled from the provided \lstinline{read_bfee.c} file. Using the \lstinline{read_bf_file} function I could store the CSI data in a variable called \textbf{\textit{csi\_trace}} which would contain every packet's CSI data in cell structures. Each cell structure contains timestamp, the number of beamforming measurements, RSS of each antenna, noise and many more. The \textbf{csi} field contains the $N_{tx}\times N_{rx}\times 30$ CSI matrix. Examining the CSI data, I can see there are a number of transmitter-receiver pair values for each subcarrier in complex number form $a + jb$. I noticed that there is some cross-talk/interference on Antenna 3 (not connected) from Antenna 1 and 2 in the NIC. The creators of the Linux CSI tool experienced this also and it is not present in the desktop computer which has all 3 antennas connected. The CSI data at this point is in Intel's internal reference level and as a result, I needed to convert the CSI to absolute units \citep{Halperin_csitool}. \par
The CSI was converted using \lstinline{get_scaled_csi()} which combines the RSSI (Received Signal Strength Indicator) and AGC (Automatic Gain Control) (gives constant output for range of inputs) to find the RSS in \lstinline{dBm} and including noise to get the SNR. I built a MATLAB script which could take all of the received packets and perform this operation in order as \lstinline{get_scaled_csi()} can only handle one cell structure at a time. This led to a CSI matrix in absolute units (voltage space) in the form $N_{tx}\times N_{rx}\times30$. Obtaining the logarithm of one packet's CSI amplitude would allow me to plot the SNR and I could plot the CSI phase over the full range of 30 subcarriers also as shown in \textbf{REFERENCE THE RESULTS}
Knowing that I was able to accurately obtain and simply process the CSI data for a $1\times 2$ and $2\times 3$ MIMO-OFDM system, I needed to implement the first stage of data processing for the amplitude and phase. 
\subsection{Data Processing}
I found that the CSI amplitude is affected by human activities as seen in WiFall \citep{WiFall}. They affect different streams independently and affect different subcarriers in a similar way. As a result, I can average the CSI amplitude samples for each stream into one value instead of 30 using the following equation and the frequency of each subcarrier around the centre frequency of $f_0 = 5GHz$:
\vspace{-11pt}
\begin{equation}\label{eqn:3.1}
    CSI_{ij}(K) = \frac{1}{N} \sum_{j=1}^{N} \frac{f_j}{f_0} |CSI_{ij}(K)|
    \vspace{-11pt}
\end{equation}
This equation was applied to the following matrix which I reconstructed in MATLAB for $K$ packets for each transmitter-receiver pair (Tx Antenna $i$ \& Rx antenna $j$):
%\vspace{-33pt}
\begin{equation}\label{eqn:3.2}
\textbf{CSI}_{ij}=\left[
\begin{array}{cccc}
    h_{ij,1}(1) & h_{ij,2}(1) & \cdots & h_{ij,30}(1) \\
    h_{ij,1}(2) & h_{ij,2}(2) & \cdots & h_{ij,30}(2) \\
   \vdots & \vdots & \ddots & \vdots \\
    h_{ij,1}(K) & h_{ij,2}(K) & \cdots & h_{ij,30}(K) 
\end{array}
\right]
\end{equation}
I then produced CSI amplitude plots using these equations/matrices to verify my methods in LOS/NLOS conditions. The CSI measured phase of a subcarrier $f$ is presented as shown as found in \cite{MonaLisaSpotFi}:
\vspace{-11pt}
\begin{equation}\label{eqn:3.3}
    \hat{\phi}_f = \phi_{f}+2\pi f_{f}\Delta t + \beta + Z_f
    \vspace{-11pt}
\end{equation}
where $\phi_{f}$ is the true phase, $\Delta t$ is time lag at the antenna, $\beta$ is an unknown constant phase offset and $Z_f$ is measurement noise. Raw true phases are unusable as the timing lag is different for each packet. The phase difference is computed as:
\vspace{-11pt}
\begin{equation}\label{eqn:3.4}
    \Delta\hat{\phi}_f = \Delta\phi_{f}+2\pi f_{f}\epsilon  + \Delta\beta + \Delta Z_f
    \vspace{-11pt}
\end{equation}
where $\Delta\phi_{f}$ is the true phase difference, $\epsilon = \Delta t1 - \Delta t2$ which are time lags at each antenna and $\Delta\beta$ is the unknown constant phase difference offset. The time lag difference is removed by placing the antennas $\frac{1}{2}\lambda$ apart at $f_{0} = 5GHz$. To mitigate random noise and outliers, I performed a linear transform on the CSI phases: 
\vspace{-11pt}
\begin{equation}\label{eqn:3.5}
    \Tilde{\phi}_f = \hat{\phi}_{f}+\frac{\phi_{N}-\phi_{1}}{k_{N}-k_1}k_{f}-\frac{1}{N}\sum_{j=1}^{N} \phi_{j}
    \vspace{-11pt}
\end{equation} \par 
According to PhaseU \citep{PhaseU}, the phase difference between two antennas is the sum of the variance of each antenna (transmitter-receiver pair) as they are both independent:
\vspace{-11pt}
\begin{equation}\label{eqn:3.6}
    \sigma^{2}_{\Delta\Tilde{\phi}_f} = \sigma^{2}_{\Delta\Tilde{\phi}_{f,1}} + \sigma^{2}_{\Delta\Tilde{\phi}_{f,2}}
    \vspace{-11pt}
\end{equation} 
The result of this linear transform on the phase difference can be seen in \textbf{LOOK FOR A PHASE DIFFERNCE PLOT WHICH SHOWS THE EFFECT}.I then carried out extensive experiments with these methods to prove that my methods were correct before performing one test where a video was also taken. The subject in the video was myself and I entered the area between the Tx and Rx about 5m apart before standing still and falling. It would be easy to notice activities/patterns in the CSI data if I could compare them against the video. From here, I spent time gathering extensive periods of data greater than 10mins for my future work.

